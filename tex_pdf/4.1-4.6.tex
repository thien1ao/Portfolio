\documentclass{article}

\usepackage{amsmath}
\usepackage[left=3cm,right=2cm,top=3cm,bottom=2cm]{geometry}
\usepackage{graphicx}
\usepackage{amssymb}
\usepackage{listings}
\usepackage{xcolor}
\usepackage{tabularx}
\usepackage{href-ul}
\usepackage{minted}
\usepackage{mathtools}
\usepackage{marvosym}



\setlength{\intextsep}{0pt}


\begin{document}
	\section*{Aufgabe 4.1}
	Mindestens eine Sechs wird gewürfelt: $P(A) = \frac{11}{36}$ \\
	\newline
	Die Summe der beiden gewürfelten Augenzahlen ist mindestens vier: $P(B) = \frac{33}{36}$ \\
	\newline
	$P(A | B) = \frac{P(A \cap B)}{P(B)} = \frac{\frac{11}{36}}{\frac{33}{36}} = \frac{1}{3} = 0.33$ \\
	\newline
	$P(B | A) = \frac{P(A \cap B)}{P(A)} = \frac{\frac{11}{36}}{\frac{11}{36}} = 1$
	
	\section*{Aufgabe 4.2}
	$P(A) = P(X=9) + P(X=10) = \binom{10}{9}(\frac{1}{2})^{10} + \binom{10}{10}(\frac{1}{2})^{10} = \frac{11}{1024}$ \\
	\newline
	$P(B) = (\frac{1}{2})^5 = \frac{1}{32}$ \\
	\newline
	$P(A \cap B) = P(\text{9  Zahlwürfe, 1 Kopfwurf}) + P(\text{10  Zahlwürfe}) = \frac{5}{1024} + \frac{1}{1024} = \frac{6}{1024}$ \\
	\newline
	$P(A|B) = \frac{\frac{6}{1024}}{\frac{1}{32}} = \frac{3}{16} = 0.1875$ \\
	\newline
	$P(B|A) = \frac{\frac{6}{1024}}{\frac{11}{1024}} = \frac{6}{11} = 0.54$
	
	\section*{Aufgabe 4.3}
	A: Weinkenner \\
	B: Nicht-Weinkenner \\
	\newline
	$P(A) = 0.03 \quad P(B) = 0.97$ \\
	\newline
	C$|$A: Weinkenner und ein Kauf wird getätigt \\
	C$|$B: Nicht-Weinkenner und ein Kauf wird getätigt \\
	\newline
	$P(C|A) = 0.85 \quad P(C|B) = 0.02$	\\
	\newline
	$P(A|C) = \frac{P(C|A)P(A)}{P(C)} = \frac{P(C|A)P(A)}{P(C|A)P(A) + P(C|B)P(B)} = \frac{0.85 \cdot 0.03}{0.85 \cdot 0.03 + 0.02 \cdot 0.97} = \frac{0.0255}{0.0449} = 0.568$
	
	\section*{Aufgabe 4.4}
	A: das Buch wird ordnungsgemäß ausgeliehen \\
	B: das Buch wird nicht ordnungsgemäß ausgeliehen \\
	\newline
	$P(A) = 0.99 \quad P(B) = 0.01$ \\
	\newline
	C$|$A: bei ordnungsgemäßer Ausleihe wird ein Alarm ausgelöst \\
	C$|$B: bei nicht ordnungsgemäßer Ausleihe wird ein Alarm ausgelöst \\
	\newline
	$P(A|C) = \frac{P(C|A)P(A)}{P(C)} = \frac{0.01 \cdot 0.99}{0.01 \cdot 0.99 + 0.95 \cdot 0.001} = 0.91 $
	\pagebreak
	\section*{Aufgabe 4.5}
	A: ausgewählte Münze ist fair \\
	B: ausgewählte Münze ist unfair \\
	\newline
	C$|$A: ausgewählte Münze wird 10-mal geworfen und sie ist fair \\
	C$|$B: ausgewählte Münze wird 10-mal geworfen und sie ist unfair \\
	\newline
	$P(W) = \frac{1}{3}$ Ziehen einer von drei Münzen \\
	\newline
	$P(C|A) = \binom{10}{8} \cdot (0.5)^8 \cdot (0.5)^2$ \\
	\newline
	$P(C|B) = \binom{10}{8} \cdot (0.8)^2 \cdot (0.2)^2$ \\
	\newline
	$P(A|C) = \frac{P(C|A)P(A)}{P(C)} = \frac{[\binom{10}{8} \cdot (0.8)^8 \cdot (0.2)^2] \cdot 0.8 }{\frac{1}{3}[\binom{10}{8} \cdot (0.5)^8 \cdot (0.5)^2]+ \frac{1}{3}[\binom{10}{8} \cdot (0.8)^8 \cdot (0.2)^2]} = \frac{0.24}{0.34} = 0.71$ \\
	\newline
	2 Münzen: $\frac{2}{3}$
	
	\section*{Aufgabe 4.6}
	$P(\text{Einschaltquote in der Gesamtbevölkerung}) = 0.11 \cdot 0.188 + 0.15 \cdot 0.553 + 0.2 \cdot 0.259 = 0.155$ \\
	\newline
	$P(unter 20 | \text{Einschaltquote in der Gesamtbevölkerung}) = \frac{0.11 \cdot 0.188}{0.155} = 0.133$
	
\end{document}