\documentclass{article}

\usepackage{amsmath}
\usepackage[left=3cm,right=2cm,top=3cm,bottom=2cm]{geometry}
\usepackage{graphicx}
\usepackage{amssymb}
\usepackage{listings}
\usepackage{xcolor}
\usepackage{tabularx}
\usepackage{href-ul}
\usepackage{minted}
\usepackage{mathtools}
\usepackage{marvosym}



\setlength{\intextsep}{0pt}


\begin{document} 
	\section*{Aufgabe 2.1}
	$P(A) = 0.87$ E-Mail \\ 
	\newline
	$P(B) = 0.83$ Facebook \\
	\newline
	$P(A \cap B) = 0.73$ beides \\
	\newline
	$P(A \cup B) = P(A) + P(B) - P(A \cap B) = 0.87 + 0.83 - 0.73 = 0.97$ \\
	\newline
	$P(A^c \cap B^c) = 1 - P(A \cup B) = 1 - 0.97 = 0.03$
	
	\section*{Aufgabe 2.2}
	$P(A \cup B) = P(A) + P(B) - P(A \cap B) = 0.8 + 0.4 - 0.3 = 0.9$ \\
	\newline 
	$P\neg(A \cup B) = 1 - P(A \cup B) = 1 - 0.9 = 0.1$ \\
	\newline
	$P\neg(A \cap B) = 1 - P(A \cap B) = 1 - 0.3 = 0.7$ \\
	\newline
	$P(A^c \cap B^c) = P(A \cup B)^c = 1 - P(A \cup B) = 1 - 0.9 = 0.1  $ \\
	\newline 
	$P(A^c \cup B^c) = P(A \cap B)^c = 1 - P(A \cap B) = 1 - 0.3 = 0.7$
	
	\section*{Aufgabe 2.3}
	\text{\large Unfairer Würfel} \\
	\newline
	keine 6 im ersten Wurf: \\
	\newline
	$P\neg(6) = 1 - P(6) = 1 - 0.2 = 0.8$ \\
	\newline
	keine 6 in beiden Würfen: \\
	\newline 
	$P\neg(6) \cdot P\neg(6) = 0.64$ \\
	\newline 
	mindestens eine 6: \\
	\newline
	$1 - P\neg(6) \cdot P\neg(6) = 0.36$ \\
	\newline 
	\text{\large Fairer Würfel} \\
	\newline
	$P(6) = \frac{1}{6} \approx 0.17 \quad P\neg(6) = 1 - P(6) = 0.83 \quad P\neg(6) \cdot P\neg(6) = 0.7 $ \\
	\newline
	mindestens eine 6: \\
	\newline
	$ 1 -P\neg(6) \cdot P\neg(6) = 1 - 0.7 = 0.3$
	\pagebreak 
	
	\section*{Aufgabe 2.4}
	\begin{figure} [h]
		\includegraphics[width=10cm]{/home/thienlao/screenshots/table.png}
	\end{figure}
	\vspace{5pt}	 	
	$E(|X-Y|) = \frac{1}{36} \sum_{i=1}^{6}\sum_{j=1}^{6}|x_{i} - y_{j}| = \frac{70}{36} = 1.94$ \\
	
	\begin{figure} [h]
		\includegraphics[width=10cm]{/home/thienlao/screenshots/table_1.png}
	\end{figure} 
	\vspace{5pt}
	$E(min(X,Y)) =  \frac{1}{36} \sum_{i=1}^{6}\sum_{j=1}^{6}min(x_{i},y_{j}) = \frac{91}{36} = 2.53 $ \\ 
	
	\begin{figure} [h]
		\includegraphics[width=10cm]{/home/thienlao/screenshots/table_2.png}
	\end{figure} 
	\vspace{5pt}
	$E(max(X,Y)) =  \frac{1}{36} \sum_{i=1}^{6}\sum_{j=1}^{6}max(x_{i},y_{j}) = \frac{161}{36} = 4.48$ 
	
	\section*{Aufgabe 2.5}
	$E(X) = 2 \cdot P(6) = \frac{1}{3} \quad E(Y) = 2 \cdot P(5) = \frac{1}{3}$ \\
	\newline
	$E(X) \cdot E(Y) = \frac{1}{9}$ \\
	\newline
	$E(X \cdot Y) = P(5,6) + P(6,5) \quad P(5,6) = P(6,5) = \frac{1}{6} \cdot \frac{1}{6} = \frac{1}{36} \implies $ \\
	\newline
	$E(X \cdot Y) = \frac{1}{18}$
	
	\section*{Aufgabe 2.6}
	1. $E(X - E(X)) =  E(X) - E(E(X)) = E(X) - E(X) = 0$ \\
	\newline
	2. $E((X - E(X))^2) = E(X^2) - (E(X))^2 \quad E(X^2) = \frac{1}{6}\sum x_{i}^2 = \frac{91}{6} \quad E(X) = \frac{1}{6}\sum x_{i} = \frac{7}{2} \implies $ \\
	\newline 
	$E((X - E(X))^2) = \frac{91}{6} - (\frac{7}{2})^2 = 2.92$ \\
	\newline
	3. $E((X - E(X))^3) = E(X^3) -3 \cdot E(X) \cdot E((X-E(X))^2 - (E(X))^3= \frac{147}{2} - 3 \cdot \frac{21}{6} \cdot \frac{35}{12} - (\frac{21}{6})^3=0$
	\pagebreak
	
	\section*{Aufgabe 2.7}
	$E(X) = (1 -P(\{(1, 1), (1, 2), (2, 1)\})) \cdot (-20) + P(\{(1, 1), (1, 2), (2, 1)\}) \cdot 1 = \frac{33}{36} \cdot (-20) + \frac{3}{36} \cdot 1 = -\frac{27}{36} = -0.75$ \\
	
	\begin{figure} [h]
		\centering
		\includegraphics[width=12cm]{/home/thienlao/screenshots/jack_gewinn.png}
	\end{figure}
	\vspace{5pt}
	Da der Erwartungswert negativ ist, sollte Jack das Spiel aufgeben
	
	\section*{Aufgabe 2.8}
	$E = P(ROT) \cdot 1 + (1 - P(ROT)) \cdot (-1) = \frac{18}{37} - \frac{19}{37} = -0.027$ \\
	\newline
	Simulationen: 100
	
	\begin{figure} [h]
		\centering
		\includegraphics[width=12cm]{/home/thienlao/screenshots/sim_2.png}
	\end{figure}
	\pagebreak
	Simulationen: 500
	\begin{figure} [h]
		\centering
		\includegraphics[width=12cm]{/home/thienlao/screenshots/sim_3.png}
	\end{figure}
	
	Simulationen: 1000
	
	\begin{figure} [h]
		\centering
		\includegraphics[width=12cm]{/home/thienlao/screenshots/sim_4.png}
	\end{figure}
	\pagebreak
	
	\section*{Aufgabe 2.9}
	Der Erwartungswert unterschiedet sich, je nachdem, ob die Person nach dem Öffnen einer Tür, bei seiner Wahl bleibt. \\
	\newline
	Wenn man seine Wahl nicht ändert, nutzt man quasi ledeglich die ursprüngliche Wahrscheinlichkeit ($P(\omega)=\frac{1}{3}$) das Geld zu gewinnen. Deshalb beträgt $E = 1 \cdot \frac{1}{3} + 0 \cdot (1 - \frac{1}{3}) = \frac{1}{3}$. Hierbei steht die 1 für das Geld und die 0 für die Ziege. \\
	\newline
	Wenn man jedoch wechselt, beträgt $E = \frac{2}{3} \cdot 1 + \frac{1}{3} \cdot 0 = \frac{2}{3}$, denn der Showmaster öffnet immer die Tür mit Ziege, d.h. die Gewinnswahrscheinlichkeit ($P(\omega) = \frac{2}{3}$). \\
	
	 \begin{figure} [h]
	 	\centering
	 	\includegraphics[width=12cm]{/home/thienlao/screenshots/anz.png}
	 \end{figure}
	
	
	
	 
	

	
	
	
	
	
		

	
		
	
\end{document}}