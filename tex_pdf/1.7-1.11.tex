\documentclass{article}

\usepackage{amsmath}
\usepackage[left=3cm,right=2cm,top=3cm,bottom=2cm]{geometry}
\usepackage{graphicx}
\usepackage{amssymb}
\usepackage{listings}
\usepackage{xcolor}
\usepackage{tabularx}
\usepackage{href-ul}


\definecolor{codegreen}{rgb}{0,0.6,0}
\definecolor{codegray}{rgb}{0.5,0.5,0.5}
\definecolor{codepurple}{rgb}{0.58,0,0.82}
\definecolor{backcolour}{rgb}{0.95,0.95,0.92}

\lstdefinestyle{mystyle}{
	backgroundcolor=\color{backcolour},   
	commentstyle=\color{codegreen},
	keywordstyle=\color{magenta},
	numberstyle=\tiny\color{codegray},
	stringstyle=\color{codepurple},
	basicstyle=\ttfamily\footnotesize,
	breakatwhitespace=false,         
	breaklines=true,                 
	captionpos=b,                    
	keepspaces=true,                 
	numbers=left,                    
	numbersep=5pt,                  
	showspaces=false,                
	showstringspaces=false,
	showtabs=false,                  
	tabsize=2
}

\lstset{style=mystyle}

\begin{document} 

		
		\section*{Aufgabe 1.7}
		
		\begin{center}
			\(
			\begin{array}{|c|c|c|c|c|c|c|}
				\hline
				\textbf{Mittelwert} & 0 & 31.7 & 32 & 32 & 32 & 32 \\
				\hline
				\textbf{Median} & 0 & 31 & 32 & 32 & 32 & 32  \\
				\hline
				\textbf{Standardabw} & 1 & 2.5 & 0.5 & 1 & 1.7 & 2.7 \\
				\hline
				\textbf{Histogramm} & 3 & 5 & 2 & 1 & 6 & 4 \\
				\hline
			\end{array}
			\)
		\end{center}
			
		\section*{Aufgabe 1.8}
		\[\bar{x}_{A}=\frac{2.2+1.8+2.1+2+2.4}{5}=2.1 \quad \sigma_{A}=\sqrt{\frac{0.2}{4}} \approx 0.22\]
		
		\[\bar{x}_{B}=\frac{5.2-2.8-3.8+4+8.4}{5}=2.2 \quad \sigma_{B}=\sqrt{\frac{111.68}{4}} \approx 5.28\]
		
			
		\section*{Aufgabe 1.9}
		\[x_{50\%}=\frac{1000}{2}=500\]
		
		\[x_{25\%}=1000 \cdot 0.25=250\]
		
		\[x_{75\%}=1000 \cdot 0.75=750\]
		
		\[\bar{x}=\frac{200 \cdot 10 + 250 \cdot 20 + 200 \cdot 30 + 200 \cdot 50 + 130 \cdot 100 + 200 \cdot 20}{1000} = 40\]
		
		\[\sigma=\sqrt{\frac{200 \cdot(10 - 40)^2 + 250 \cdot(20-40)^2+ 200 \cdot (30 -40)^2+ 200\cdot(50-40)^2+130\cdot (100-40)^2+20\cdot(200-40)^2}{1000}}\]
		
		\[=\sqrt{1300}=36\]
		
		\section*{Aufgabe 1.10}
		\[x_{10\%}=200\cdot0.1=20 \quad \text{liegt bei Alter 5}\] 
		
		\[x_{50\%}=200\cdot0.5=100 \quad \text{liegt bei Alter 8} \]
		
		\[x_{90\%}=200\cdot0.9=180 \quad \text{liegt bei Alter 13} \]
		
		\section*{Aufgabe 1.11}
		DAX shwankt stärker an Dienstagen, weil Maximawerte dienstags größer (bzw. Minimawerte kleiner) sind als montags.
	
\end{document}}