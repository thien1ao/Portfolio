\documentclass{article}

\usepackage{amsmath}
\usepackage[left=3cm,right=2cm,top=3cm,bottom=2cm]{geometry}
\usepackage{graphicx}
\usepackage{amssymb}
\usepackage{listings}
\usepackage{xcolor}
\usepackage{tabularx}
\usepackage{href-ul}
\usepackage{minted}
\usepackage{mathtools}
\usepackage{marvosym}



\setlength{\intextsep}{0pt}


\begin{document}
		\section*{Aufgabe 3.1}
		\begin{figure} [h]
			\centering
			\includegraphics[width=8cm]{/home/thienlao/screenshots/coin.png}
		\end{figure}
		\vspace{2cm}
		\begin{figure} [h]
			\centering
			\includegraphics[width=8cm]{/home/thienlao/screenshots/wahr_coin.png}
		\end{figure}
		\vspace{2cm}
		\begin{figure} [h]
			\centering
			\includegraphics[width=8cm]{/home/thienlao/screenshots/box_coin.png}
		\end{figure}
		
		\section*{Aufgabe 3.2}
		a) $\frac{(\frac{4}{2})(\frac{28}{2})}{(\frac{32}{4})} = \frac{2268}{35960} \approx 0.063$ - W.keit 2 Asse  \\
		\newline
		b) $\frac{(\frac{24}{6})}{(\frac{49}{6})} \approx 0.0096$ - W.keit 6 aus 49 sind gerade Zahlen
		
		\section*{Aufgabe 3.3}
		$\frac{(\frac{100 - 3}{50 - 3})}{(\frac{100}{50})} \approx 0.1212$ - W.keit, dass 3 Freunde zusammen bleiben
		
		\section*{Aufgabe 3.4}
		$P = \frac{1}{10} \quad \neg P = 1 - P = \frac{9}{10}$ - Schlüssel wird falsch gezogen \\
		\newline 
		W.keit nach 20 Versuche $(\frac{9}{10})^{20} \approx 0.1216$
		
		\section*{Aufgabe 3.5}
		n Würfe Kopf: $(\frac{1}{2})^n \implies$ mindestens einmal Zahl $1  - (\frac{1}{2})^n$ \\
		\newline
		$1 - (\frac{1}{2})^n \geq 0.95 \implies$ \\
		\newline
		$(\frac{1}{2})^n \leq 0.05 \implies$ \\
		\newline
		$\log(\frac{1}{2})^n \leq \log(0.05) \implies$ \\
		\newline
		$n \geq \frac{\log0.5}{\log0.05} \approx 4.35 \implies n=5$
		
		\section*{Aufgabe 3.6}
		$(\frac{18}{2}) = 153$ - Gesamtzahl der Sitzkombinationen für 2 Personen \\
		\newline
		15 (3 Reihen je 5 Plätze) $\implies$ W.keit Romeo und Julia sind Nachbarn $=\frac{15}{153} \approx 0.098$ \\
		\newline
		$P=\frac{1}{18}$ - W.keit, mit welcher Romeo  seinen Platz einnimmt \\
		\newline
		12 - Möglichkeiten 2 Nachbarn;  6 - Möglichkeiten 1 Nachbar $\implies \frac{1}{18}(12 \cdot \frac{1}{17} + 6 \cdot \frac{1}{17}) \approx 0.098$
		
		\section*{Aufgabe 3.7}
		\begin{figure} [h]
			\includegraphics[width=4cm]{/home/thienlao/screenshots/quant.png}
		\end{figure}
		
		\section*{Aufgabe 3.8}
		\(
		\begin{array}{|c|c|c|c|c|c|c|c|}
			\hline
			x_{i} & 0 & 1 & 2 & 3 & 4 \\
			\hline
			P(X=x_{i}) & 0.07 & 0.3535 & 0.4242 & 0.1414 & 0.0101 \\
			\hline
		\end{array}
		\) \\
		\vspace{5pt}
		\newline
		$P(X=0)=\frac{\binom{5}{0}\binom{7}{4}}{\binom{12}{4}}=0.07$ \\
		\newline
		$P(X=1)=\frac{\binom{5}{1}\binom{7}{3}}{\binom{12}{4}}=0.3535$ \\
		\newline	$P(X=2)=\frac{\binom{5}{2}\binom{7}{2}}{\binom{12}{4}}=0.4242$ \\
		\newline
		$P(X=3)=\frac{\binom{5}{3}\binom{7}{1}}{\binom{12}{4}}=0.1414$ \\
		\newline
		$P(X=4)=\frac{\binom{5}{4}\binom{7}{0}}{\binom{12}{4}}=0.0101$ 
		
		\section*{Aufgabe 3.9}
		Mögliche Kombinationen: \{SSS, SLS, SSL, LSS, LSL\} \\
		\newline 
		Man muss zwei aufeinanderfolgende
		Klausuren bestehen: $b_{1} \cdot b_{2} \quad b_{2} \cdot b_{3} \quad b_{1} \cdot b_{2} \cdot b_{3}$ wobei $b_{1}, b_{2}, b_{3}$ \\
		- bestandene Klauseren \\
		\newline 
		$P(A \cup B) = P(A) + P(B) - P(A \cap B) \implies $ \\
		\newline
		$P(SSS) = p_{1} \cdot p_{1} + p_{1} \cdot p_{1} - p_{1} \cdot p_{1} \cdot p_{1} = 2p_{1}^2 - p_{1}^3 $ \\
		\newline
		$P(SLS) = p_{1} \cdot p_{2} + p_{2} \cdot p_{1} - p_{1} \cdot p_{2} \cdot p_{1} = 2p_{1}p_{2} - p_{1}^2p_{2} $ \\
		\newline
		$P(SSL) = p_{1} \cdot p_{1} + p_{1} \cdot p_{2} - p_{1} \cdot p_{1} \cdot p_{2} = p_{1}^2 + p_{1}p_{2} - p_{1}^2p_{2} $ \\
		\newline
		$P(LSS) = p_{2} \cdot p_{1} + p_{1} \cdot p_{1} - p_{23} \cdot p_{1} \cdot p_{1} = p_{1}^2 + p_{1}p_{2} - p_{1}^2p_{2} $ \\
		\newline
		$P(LSL) = p_{2} \cdot p_{1} + p_{1} \cdot p_{2} - p_{2} \cdot p_{1} \cdot p_{2} = 2p_{1}p_{2} - p_{2}^2 \cdot p_{1} $ \\
		\newline
		$P(SSS) < P(SSL) < P(SLS) < P(LSL) \implies$  Reihenfolge LSL (leichte$->$schwere$->$leichte)
		
\end{document}}